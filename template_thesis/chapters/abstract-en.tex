Frontend applications are an essential part of any web application and are responsible for parts of the business logic and the UI. Increasing demands for speed, responsiveness, and features significantly raise complexity. To manage part of this complexity, state management solutions are used. These handle important tasks such as data fetching, data transformation, and data storage. Errors in the state can, therefore, have a relatively large impact on the user experience and operational business. To reduce and quickly detect errors and defects in this area, a strict extension for state management, in general, is introduced and compared to the conventional approach. A reduction in error-proneness as well as improvements in developer experience, readability, and maintainability are considered plausible, even though these aspects could not be definitively proven within the scope of the study.