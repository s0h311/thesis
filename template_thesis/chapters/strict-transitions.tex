\chapter{Strict Transitions} \label{ch:strict-transitions}

In diesem Kapitel wird der \acrlong{ac:st}-Ansatz vorgestellt. Außerdem werden zwei Beispiele für die Implementierung von \acrshort{ac:st} für jeweils Redux und Pinia auf Basis der Erkenntnisse aus dem voherigen Kapitel aufgeführt.

\section{Steigende Robustheit durch TypeScript}

% Mehr Projekte nutzen TypeScript => sind bereits was Typing angeht robust
% TypeScript ist auch in fast allen IDEs integriert, also wird fast von jedem genutzt.
\acrlong{ac:ts} verfügt, im Gegensatz zu JavaScript, über statische Typisierung. Dank der statischen Typisierung sind statische Typeanalysen und Operationen wie \textit{Go to Definition} und \textit{Go to Implementations} der \acrlong{ac:ide}s (\acrshort{ac:ide}) möglich. Diese Eigenschaften reduzieren Fehler im Zusammenhang mit falschen Typen erheblich. Wie in \ref{fig:js-und-ts-nutzung-2018-2024} abgebildet, wird \acrlong{ac:ts} von immer mehr Entwicklern genutzt, während die JavaScript-Nutzung abnimmt. \ref{fig:js-und-ts-nutzung-2018-2024} beinhaltet die tatsächliche Nutzung von \acrshort{ac:ts} nicht. Der TypeScript Compiler ist in den meisten modernen \acrshort{ac:ide}s, wie Visual Studio Code und den JetBrains \acrshort{ac:ide}s wie IntelliJ IDEA und WebStorm integriert. Dies führt dazu, dass man auch beim JavaScript-Code einige Vorteile von \acrlong{ac:ts} bekommt.\cite{typeScriptDocumentary}

\begin{figure}[H]
  \includegraphics[width=0.85\textwidth]{js-und-ts-nutzung-2018-2024.png}
  \caption{Stack Overflow Surveys der Jahre
  [\citeyear{stackOverflowSurvey2018MostPopularTechnologies}],
  [\citeyear{stackOverflowSurvey2019MostPopularTechnologies}],
  [\citeyear{stackOverflowSurvey2020MostPopularTechnologies}],
  [\citeyear{stackOverflowSurvey2021MostPopularTechnologies}],
  [\citeyear{stackOverflowSurvey2022MostPopularTechnologies}],
  [\citeyear{stackOverflowSurvey2023MostPopularTechnologies}] und
  [\citeyear{stackOverflowSurvey2024MostPopularTechnologies}], Prozentuale Nutzung von JavaScript und \acrlong{ac:ts} unter professionellen Entwicklern von 2018 bis 2024}
  \label{fig:js-und-ts-nutzung-2018-2024}
\end{figure}

% Um Implementierungsfehler zu vermeiden => Transitions korrekter machen

\section{Fehlende Garantie für Korrektheit des States}

Der \acrlong{ac:ts}-Faktor macht Webapplikationen, somit auch \acrlong{ac:sm}, auf Typ-Ebene robuster und weniger fehleranfällig. Allerdings ist es für die Applikation immer noch möglich, in einem falschen Zustand zu sein. Gegeben sei ein Redux Store, der für das Speichern einer Liste von \textit{items} zuständig ist. Definiert wird der Store wie folgt:

\begin{lstlisting}
  type FetchAction = {
    type: 'fetch'
  }

  type FetchSuccessfulAction = {
    type: 'fetch-successful',
    payload: Array<any>
  }

  type FetchFailedAction = {
    type: 'fetch-failed',
    payload: Error
  }

  type Action =
    | FetchAction
    | FetchSuccessfulAction
    | FetchFailedAction

  function reducer(
    state = { items: 'not-fetched' }, 
    action: Action
  ) {
    switch (action.type) {
      case 'fetch':
        return {
          ...state,
          items: 'fetching'
        }
      case 'fetch-successful':
        return {
          ...state,
          items: action.payload
        }
      case 'fetch-failed':
        return {
          ...state,
          items: action.payload
        }
      default:
        return state
    }
  }
  
  const store = createStore(reducer)
\end{lstlisting}

Es ist erlaubt, die \textit{FetchSuccessfulAction} Action zu versenden, ohne vorher die \textit{Fetch} Action versendet zu haben. Das heißt: \q{\textit{items} wurden erfolgreich abgerufen}, ohne die Anfrage zuvor gemacht zu haben. Seitens Redux ist das Versenden einer Action immer, unbeachtet des aktuellen Zustandes, erlaubt. Dieser Faktor spricht gegen die Nachvollziehbarkeit und gilt für alle populären \acrshort{ac:sm}-Lösungen.

\section{Korrektere Zustandsübergänge}
Es wird vorgeschlagen, den Applikationszustand wie einen Zustand eines \acrshort{ac:dfa}s zu behandeln. Im Falle von Redux werden die Actions als Übergänge und der Reducer als die Übergangsfunktion eines \acrshort{ac:dfa}s gesehen. Restliche Eigenschaften des Quintupels eines \acrshort{ac:dfa}s werden hierbei ignoriert.

Um die Übergangsfunktion zu definieren, wird pro Zustand eine Liste aller Actions (\textit{Übergangsliste}) benötigt, die bei diesem Zustand erlaubt sind. Ein Problem hierbei ist allerdings, dass die Identifizierbarkeit der einzelnen Zustände nicht garantiert ist. Abweichend von \acrshort{ac:dfa}s sind die Zustände in Webapplikationen nicht immer serialisierbar. Nicht serialisierbare Objekte sind nicht immer identifizierbar. Im oberen Beispiel sind die Zustände \textit{'not-fetched'} und \textit{'fetching'} vom Typ String und somit serialisierbar, allerdings sind die restlichen Zustände nicht serialisierbar (Zustand vom Typ Error und Array<any>). Um dieses Problem zu umgehen, wird eine \textit{Identitätsfunktion} empfohlen, um zwischen verschiedenen Zuständen zu unterscheiden. Sie akzeptiert als Parameter den aktuellen Zustand und gibt ein Boolean zurück.

\begin{lstlisting}
type IdentityFn<S> = (state: S) => boolean
\end{lstlisting}

% ODER SO:
% Sei $S$ die Menge aller Zustände und $B := \{true, false\}$, dann wird die \textit{Konfition Funktion} wie folgt definiert:
% $f(s) = b | s \in S \land b \in B$

Mit dieser Funktion kann der Anwender für die Identifizierbarkeit der Zustände sorgen. Bei JavaScript-Klassen kann der \textit{instanceof}-Operator genutzt werden, um auf die Instanz einer Klasse wie \textit{Error} zu prüfen.\cite{jsInstanceofOperator} Des Weiteren können bei Objekten auf eindeutige Eigenschaften, wie die Präsenz eines Feldes per \textit{in}-Operator geprüft werden.\cite{jsInOperator} Bei Arrays kann die \textit{Array.isArray}-Funktion verwendet werden.\cite{jsIsArray} Durch die Kombination dieser und weiterer Funktionen und Operatoren können weitere Datentypen und Fälle identifiziert werden.

Die Übergangsliste lässt sich in einer Map wie folgt speichern:

\begin{lstlisting}
type TransitionMap<S extends IdentityFn<S>, A> = Map<S, Array<A>>
\end{lstlisting}

Für die Transition Map gilt: Identitätsfunktion ist der Schlüssel, während die Liste von Actions der Wert ist.

In der Übergangsfunktion wird der Zustandswechsel mit einer \textit{Validierungsfunktion} validiert. Diese prüft mit Hilfe der Transition Map auf die Gültigkeit des Übergangs und wirft einen Laufzeitfehler bei ungültigen Aufrufen. Falls der Übergang gültig ist, wirft sie keinen Fehler und der Zustandswechsel kann stattfinden. Gültig ist der Übergang, wenn es für den aktuellen Zustand eine Identitätsfunktion gibt, die wahr zurückgibt und die aktuelle Action in der zugehörigen Liste enthalten ist. In allen anderen Fällen ist der Übergang ungültig. Der Laufzeitfehler sorgt dafür, dass der ungültige Aufruf berichtet wird und sich nicht zu einem langlebigen Bug entwickeln kann.

% TODO stimmt nicht: "In der Übergangsfunktion wird der Zustandswechsel mit einer \textit{Validierungsfunktion} validiert."

Die Validerungsfunktion ist wie folgt definiert:

\begin{lstlisting}
type ValidationFn<S, A> = (
    transitionMap: TransitionMap<S, A>,
    state: S,
    action: A
  ) => boolean
\end{lstlisting}

\subsection{Vorteile}

\begin{enumerate}
  \item \textbf{Übersichtlichkeit}: Damit die Validierung funktioniert, ist der Entwickler gezwungen, die Transition Map zu definieren. So können fehlerhafte und überflüssige Übergänge schneller erkannt und korrigiert werden.
  \item \textbf{Erkennung von Bugs}: Bei fehlgeschlagener Übergangsvalidierung wird ein Laufzeitfehler geworfen, der über die Monitoringsysteme die Entwickler über einen Bug informieren kann. Ebenfalls möglich ist es, die falschen Übergänge lediglich zu loggen. Auf diesem Wege können die Entwickler ebenfalls über den Bug in Kenntnis gesetzt werden. Die letztere Strategie erlaubt jedoch im Worst-Case die Weiterausführung falscher Geschäftslogik.
\end{enumerate}

\subsection{Nachteile}

\begin{enumerate}
  \item \textbf{Mehr Aufwand}: Damit der Ansatz funktioniert, muss die Transition Map definiert werden. Diese Voraussetzung kostet zusätzlichen Aufwand.
  \item \textbf{Erhöhte Ausführungszeit}: Außerdem erhöht sich die Ausführungszeit der gesamten Applikation durch die Validierung bei jedem Zustandswechsel. Diese hinzukommende Zeit ist jedoch zu vernachlässigen, wenn die Identitätsfunktion effizient ist und keine Nebeneffekte erzeugt, also eine Pure-Function ist.
\end{enumerate}

\section{Implementierung}

\subsection{Redux}

Im Folgenden wird die Implementierung der oben genannten Funktionen und Konzepte für Redux gezeigt.

Die Transition Map ist die Grundlage des Ansatzes. Mit Blick auf die \acrlong{ac:dx} und die Lesbarkeit wird die Transition Map als ein Array von Objekten mit zwei Feldern definiert, nämlich \textit{identityFn} und \textit{actionTypes}:

\begin{lstlisting}
type Transition<S> = {
  identityFn: (state: S) => boolean
  actionTypes: string[]
}
  
type Transitions<S> = Transition<S>[]
\end{lstlisting}

Beispiel für eine Transition Map:

\begin{lstlisting}
type State = 'not-fetched' | 'fetching' | string[] | Error

type Action =
  | {
      type: 'fetch'
    }
  | {
      type: 'fetch-successful'
      payload: string[]
    }
  | {
      type: 'fetch-failed'
      payload: Error
    }
  
const transitions = [
  {
    identityFn: (state) => state === 'not-fetched',
    actionTypes: ['fetch'],
  },
  {
    identityFn: (state) => state === 'fetching',
    actionTypes: ['fetch-successful', 'fetch-failed'],
  },
]
\end{lstlisting}

Der Reducer kann wie von Redux vorgegeben definiert werden:

\begin{lstlisting}
function reducer(state = 'not-fetched', action) {
  switch (action.type) {
    case 'fetch':
      return 'fetching'
    case 'fetch-successful':
      return action.payload
    case 'fetch-failed':
      return action.payload
    default:
      return state
  }
}
\end{lstlisting}

Die definierten Übergänge werden mit Hilfe der folgenden Validierungsfunktion validiert:

\begin{lstlisting}
function validateTransition(state, action, transitions) {
  for (const transition of transitions) {
    if (transition.identityFn(state)) {
      if (transition.actionTypes.includes(action.type)) {
        return
      }
  
      throw new IllegalTransitionError(state, action.type)
    }
  }
  
  throw new TransitionNotFoundError(state)
}
\end{lstlisting}

Die beiden Laufzeitfehler IllegalTransitionError und TransitionNotFoundError erben von der Error-Klasse und dienen der Unterscheidbarkeit.

Die Validierungsfunktion wird vor der \textit{Store.dispatch} Methode ausgeführt. Damit dies gelingt, wird die folgende Funktion definiert und zum Versenden von Actions genutzt.

\begin{lstlisting}
function dispatchTransition(this, action) {
  validateTransition(this.getState(), action, this.transitions)
  
  this.dispatch(action)
}
\end{lstlisting}

Es wird ein Proxy für \textit{createStore} eingeführt, der wie folgt implementiert ist:

\begin{lstlisting}
function createTransitionStore<S>(
  transitions: Transitions<S>,
  ...args: Parameters<typeof createStore>
): TransitionStore<S> {
  const store = createStore(...args)
  
  Object.defineProperty(store, 'transitions', {
    value: transitions,
  })
  
  Object.defineProperty(store, 'validateTransition', {
    value: validateTransition,
  })
  
  Object.defineProperty(store, 'dispatchTransition', {
    value: dispatchTransition,
  })
  
  return store
}
\end{lstlisting}

Die \textit{createTransitionStore}-API mit dem zusätzlichen \textit{transitions} Parameter erstellt einen Store mit der \textit{createStore}-API von Redux und fügt dem Store drei neue Felder hinzu: die Transition Map, die Validierungsfunktion und die dispatchTransition-Methode.

\begin{lstlisting}
type TransitionStore<S, A> = {
  validateTransition: typeof validateTransition
  transitions: Transitions<S>
  dispatchTransition: typeof dispatchTransition
} & Store  
\end{lstlisting}

Der TransitionStore kann per \textit{createTransitionStore} erstellt werden. Die Actions werden per \textit{TransitionStore.dispatchTransition(action)} dispatched. Mit dieser Implementierung bleiben die APIs zum Lesen und Manipulieren des Stores identisch. Lediglich die Funktion zur Erstellung eines Stores nimmt einen zusätzlichen \textit{transitions} Parameter.

\subsection{Pinia}

Pinia verfügt über ein Plugin-System. Über dieses erhält man unter anderem Zugriff auf den Zustand und die Actions der aktiven Stores. Die Plugins werden beim Start der Applikation über die \textit{Pinia.use} API registriert.\cite{piniaPluginSystem}

Die Transition Map hat die gleiche Struktur wie die bei Redux. Allerdings wird das Feld \textit{actionTypes} zu \textit{action} umbenannt und steht für den Namen der Action im \textit{actions}-Objekt.

Das Plugin wird mit der \textit{transitions}-Funktion instanziiert, diese nimmt eine Map mit der ID des Stores als Schlüssel und die Transition Map als Wert für den jeweiligen Store:

\begin{lstlisting}
type PiniaUseCallback =
  Parameters<ReturnType<typeof createPinia>['use']>[0]

type transitions<S> = (
  transitionsByStoreId: TransitionsByStoreId<S>
) => PiniaUseCallback

type TransitionsByStoreId<S> = {
  [storeId: string]: Transitions<S>
}
\end{lstlisting}

Die \textit{transitions}-Funktion gibt eine anonyme Funktion zurück, die ein Objekt als Parameter erhält. In diesem befindet sich unter anderem das Store-Objekt. Über die \textit{Store.\$onAction}-Methode kann eine Callback-Funktion als Preprocessor für Actions registriert werden. Die Callback-Funktion erhält ein Objekt als Parameter. In diesem sind unter anderem der Name der aktuellen Action und der aktuelle Zustand enthalten. In dem Preprocessor wird die Validierungsfunktion aufgerufen.

\begin{lstlisting}  
type PiniaUseCallbackArgs = Parameters<PiniaUseCallback>[0]

function transitions<S>(
  transitionsByStoreId: TransitionsByStoreId<S>
): PiniaUseCallback {
  return ({ store }: PiniaUseCallbackArgs) => {
    const transitions = transitionsByStoreId[store.$id]
  
    if (transitions !== undefined) {
      store.$onAction(({ name, store }) => {
        // name ist der Name der aktuellen Action
        validateTransition(store.$state, name, transitions)
      })
    }
  }
}  
\end{lstlisting}

Die Validierungsfunktion ist auf die leicht geänderte Struktur der Actions angepasst:

\begin{lstlisting}
function validateTransition<S, A extends string>(
  state: S,
  action: A,
  transitions: Transitions<S>
): void {
  for (const transition of transitions) {
    if (transition.identityFn(state)) {
      if (transition.actions.includes(action)) {
        return
      }

      throw new IllegalTransitionError(state, action)
    }
  }

  throw new TransitionNotFoundError(state)
}
\end{lstlisting}

Das Plugin wird im Bootstrap-Schritt der Vue-Applikation registriert:

\begin{lstlisting}
const app = createApp(App)
const pinia = createPinia()

pinia.use(stateTransitions({
  [itemStoreId]: itemStoreTransitions
}))

app.use(pinia)
app.mount('#app')
\end{lstlisting}

Im Gegensatz zum TransitionStore für Redux ändern sich bei Pinia, dank des Plugin-Systems, die APIs zur Erstellung, zum Lesen und zur Manipulation der Stores nicht.
