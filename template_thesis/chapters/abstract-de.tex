Frontend Applikationen sind ein wesentlicher Bestandteil jeder Webapplikation und sind für Teile der Geschäftslogik und die UI verantwortlich. Steigende Anfordungen in Geschwindigkeit, Responsiveness und Features erhöhen die Komplexität enorm. Um ein Teil dieser Komplexität zu verwalten, kommen State Management Lösungen zum Einsatz. Diese übernehmen wichtige Aufgaben wie beispielsweise das Data Fetching, Datentransformation und die Datenspeicherung. Fehler im State kann daher einen verhältnismäßig großen Einfluss auf das Nutzererlebnis und das operative Geschäft haben. Damit Fehler und Defekte in diesem Bereich reduziert und schnell erkannt werden, wird eine strikte Erweiterung für State Management im Allgemeinen vorgestellt und mit dem normalen Ansatz verglichen.