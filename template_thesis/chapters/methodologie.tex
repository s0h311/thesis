\chapter{Methodologie}

In dieser Arbeit werden die bestehende SM-Ansätze um Übergänge wie bei einem DFA erweitert. Um dies zu erreichen, ist es notwendig die Funktionsweise bestehender Ansätze zu kennen. Diese werden im Kapitel \ref{sm-ansaetze} aufgeführt. Anschließend werden die DFA-Übergänge angepasst auf Anwendungfall einschließlich der JavaScript API zur Definition im Kapitel \ref{der-neue-ansatz} dargestellt. Die Erkenntnisse aus Kapitel \ref{sm-ansaetze} und \ref{der-neue-ansatz} werden kombiniert, um zwei konkrete Implementierungen für redux und pinia zu zeigen. Abschließend werden diese im Kapitel \ref{vergleich} verglichen.

\section{Vorstellung bestehender Ansätze}



\section{Code Ausschnitte}

TypeScript wird benutzt um, Aufbau von Objekten oder Funktionen zu beschreiben. Längere Strukturen werden mit Hilfe von Code-Bespielen veranschaulicht. Hierfür wird ebenfalls TypeScript verwendet. An viele Stellen wird auf Type-Annotationen verzichtet, damit die Beispiele leicht lesbar bleiben. Es wird auf Semikolon verzichtet, da diese bei JavaScript nicht erforlich sind.