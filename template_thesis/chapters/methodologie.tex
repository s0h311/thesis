\chapter{Methodologie}

% 1. Bestehende State-Management-Lösungen/Ansätze vorstellen, Funktionsweise
%   -> Dabei Code-Snippets nutzen
% 2. Welches Problem gibt es?
% 3. Wie löst der ST-Ansatz das Problem?
% 4. Implementierung des Ansatzes mit Code-Snippets
% 5. Vergleich der Aspekte
%   -> Dabei quantitative und qualitative Aspekte
%   -> Quantitative Aspekte basieren auf üblichen Coding-Standards, Konventionen und Konzepten
% 6. Beantwortung der Leitfrage auf Basis des Vergleichs

In dieser Arbeit kamen quantitative Methoden zur Analyse von \acrlong{ac:loc} (\acrshort{ac:loc}), Bundle Size und Performance zum Einsatz. Zur Analyse der \acrlong{ac:dx}, Fehleranfälligkeit, Lesbarkeit und Wartbarkeit wurden hingegen qualitative Methoden verwendet. Die Vier Aspekte Fehlerquote, \acrshort{ac:dx}, Lesbarkeit und Wartbarkeit, mit denen sich die Forschungsfragen beschäftigen, sind weitgehend mithilfe qualitativer Methoden beantwortbar. Da der quantitative Aspekt der \acrshort{ac:loc} auch zu den genannten qualitativen Aspekten beiträgt, wird er in dieser Arbeit inkludiert. Die ebenfalls berücksichtigten quantitativen Merkmale der Performance und Bundle Size sind vor allem im Web von hoher Bedeutung und korrelieren mit wichtigen wirtschaftlichen Kennzahlen wie der Conversion-Rate.\cite{googleConversionRateSpeed}

% TODO der Satz: "Die Vier Aspekte Fehlerquote, \acrshort{ac:dx}, Lesbarkeit und Wartbarkeit, mit denen sich die Forschungsfragen beschäftigen, sind weitgehend mithilfe qualitativer Methoden beantwortbar." ist nicht zutreffen und ist irreführend. ÜBERARBEITEN

\section{Aufbau}

Damit die Forschungsfragen beantwortet werden können, wird der \acrshort{ac:st}-Ansatz zum normalen Ansatz verglichen. Damit dieser möglichst realitätsnah ist, wurde eine Webapplikation mit üblichen Anforderungen wie Data-Fetching und Filterung gebaut. Die Applikation besteht aus einer Seite, welche eine Liste von 194 Produkten und Filteroptionen beinhaltet. Die Applikation fetcht und speichert die Produktdaten im Store. Die Filteroptionen sind: Titel, Preisobergrenze, Mindestbewertung, Verfügbarkeit und Kategorie.

Die Applikation wurde in React mit Redux und analog in Vue mit Pinia gebaut. Anschließend wurden die beiden Implementierungen kopiert und die Kopien um \acrlong{ac:st} erweitert.

Das öffentliche GitHub-Repository mit allen Applikationen ist im Anhang \ref{ap:githubRepository} verlinkt.

\subsection{Quantitative Methoden}
Die quantitativen Kennzahlen wurden mithilfe des Unix-Utilitys \textit{wc} (word count) für LOC, Playwright und Chrome-Profiling für Performance sowie dem Build-Tool Vite für Bundle Size gesammelt. Das Ergebnis der Analysen sind die relativen und absoluten Veränderungen der Kennzahlen.

\subsection{Qualitative Methoden}
Die qualitative Analyse basiert auf weitverbreiteten Code-Konventionen, Patterns und Empfehlungen, die zur Lesbarkeit und Wartbarkeit des Quellcodes beitragen und die Produktivität des Entwicklers sowie die Fehlerquote der Applikation unmittelbar beeinflussen. Es ist wichtig zu erwähnen, dass die Schlussfolgerungen hierbei nicht vollständig von Subjektivität befreit sind.

\section{Code-Ausschnitte}

\acrlong{ac:ts} (\acrshort{ac:ts}) wird benutzt, um den Aufbau von Objekten oder Funktionen zu beschreiben. Längere Strukturen werden mithilfe von Code-Beispielen veranschaulicht. Hierfür wird ebenfalls \acrlong{ac:ts} verwendet. An vielen Stellen wird auf Type-Annotationen und Import-Statements verzichtet, damit die Beispiele leicht lesbar bleiben.
