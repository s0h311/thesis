\chapter{Methodologie}

% 1. Bestehende State Management Lösungen/Ansätze vorstellen, Funktionsweise
%   -> Dabei Code Snippets nutzen
% 2. Welches Problem gibt es?
% 3. Wie löst der ST Ansatz das Problem
% 4. Implementierung des Ansatzes mit Code Snippets
% 5. Vergleich der Aspekte
%   -> Dabei quantitative und qualitative Aspekte
%   -> Quantitative Aspekte basieren auf übliche Coding Standards, Konventionen und Konzepte
% 6. Beantwortung der Leitfrage auf Basis des Vergleichs



\section{Code Ausschnitte}

\acrlong{ac:ts} wird benutzt um, Aufbau von Objekten oder Funktionen zu beschreiben. Längere Strukturen werden mit Hilfe von Code-Bespielen veranschaulicht. Hierfür wird ebenfalls \acrlong{ac:ts} verwendet. An viele Stellen wird auf Type-Annotationen verzichtet, damit die Beispiele leicht lesbar bleiben.