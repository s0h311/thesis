\chapter{Vergleich der Ansätze} \label{vergleich}

Der in vorangegangen Kapiteln vorgestellte \acrlong{ac:st} Ansatz, erweitert die interne Funktionsweise einer \acrlong{ac:sm}-Lösung. Er erfordert die zusätzliche Definition einer TranisitionMap und es werden leicht geänderte APIs dem Benutzer zur Verfügung gestellt. In diesem Kapitel werden die Standard Stores (unveränderte) zu den mit \acrlong{ac:st} verglichen. Es werden die quantifizierbare Kennzahlen Lines of Code, Bundle Size und Performance untersucht. Außerdem werden die Aspekte \acrlong{ac:dx}, Fehleranfälligkeit, Wartbarkeit und Lesbarkeit analysiert.

\section{Quantifizierbare Aspekte}
\subsection{Lines of Code}

Bei React wurden die .js, .ts, .tsx und bei Vue die .ts und .vue Dateien untersucht. Alle anderen Dateitypen wurden ignoriert, da diese weder für die Geschäfslogik noch für die UI verantwortlich sind.

\begin{table}[!ht]
  \caption{Statische Analyse bei Redux und Pinia mit und ohne \acrshort{ac:st}}
  \label{tab:staticAnalysisSTvsNoST}

  \begin{center}
    \begin{tabular}{|c|c|c|c|} 
    \hline
    Dateityp & Anzahl & LOC & Szenario \\ [0.5ex] 
    \hline\hline
    js & 1 & 25 & React ohne ST \\ 
    \hline
    ts & 14 & 256 & React ohne ST \\
    \hline
    tsx & 5 & 187 & React ohne ST \\
    \hline\hline
    js & 1 & 25 & React mit ST \\ 
    \hline
    ts & 17 & 284 & React mit ST \\
    \hline
    tsx & 5 & 187 & React mit ST \\
    \hline\hline
    ts & 5 & 140 & Vue ohne ST \\
    \hline
    vue & 5 & 133 & Vue ohne ST \\
    \hline\hline
    ts & 5 & 172 & Vue mit ST \\
    \hline
    tsx & 5 & 133 & Vue mit ST \\
    \hline
    \end{tabular}
  \end{center}
\end{table}

Der relative Anstieg in \acrshort{ac:loc} beträgt bei React $\sim6\%$ und bei Vue $\sim12\%$. Darüber hinaus beträgt der absolute Anstieg jeweils 28 Lines und 32 Lines. Siehe (Tab. \ref{tab:staticAnalysisSTvsNoST})

\subsection{Bundle Size}

Analysiert wurden die Production Bundles der Applikationen. Diese wurden mit dem Build-Tool Vite erstellt. Alle Applikationen nutzen ausschließlich Client Side Rendering. Daher ist es wichtig, dass die Bundles klein wie möglich bleiben.

\begin{table}[!ht]
  \caption{Bundle Size Analyse bei Redux und Pinia mit und ohne \acrshort{ac:st}}
  \label{tab:bundleSizeAnalysisSTvsNoST}

  \begin{center}
    \begin{tabular}{|c|c|c|} 
    \hline
    Size in kB & Gziped & Szenario \\ [0.5ex]
    \hline\hline
    156,26 & 51,22 & React ohne ST \\
    \hline
    157,24 & 51,54 & React mit ST \\
    \hline
    70,43 & 28,23 & Vue ohne ST \\
    \hline
    71,16 & 28,5 & Vue mit ST \\
    \hline
    \end{tabular}
  \end{center}
\end{table}

Der relative Anstieg beträgt bei React $\sim0,6\%$ und bei Vue $\sim1\%$. Darüber hinaus beträgt der absolute Anstieg jeweils 0,98kB und 0,73kB. Der Anstieg in Bundle Size ist vernachlässigbar. Siehe (Tab. \ref{tab:bundleSizeAnalysisSTvsNoST})

\subsection{Performance}

Um den Unterschied in Performance zu messen, wurde das Testing Tool Playwright verwendet. Mit Hilfe von Playwright wurde ein Szenario definiert. In diesem wurden alle Features der Webseite verwendet, welche Actions in Stores verursachen. Das Szenario wurde pro Applikation 20 Mal ausgeführt. Es lief im Chrome Browser und wurde mit Hilfe des Performance Tabs in den Chrome DevTools analysiert. Es wurden die Mittelwerte für Ausführungszeit in Millisekunden für die Browsertasks Scripting, Painting und Rendering ermittelt.

\begin{table}[!ht]
  \caption{Performance Analyse Redux und Pinia mit und ohne \acrshort{ac:st}}
  \label{tab:performanceAnalysisSTvsNoST}

  \begin{center}
    \begin{tabular}{|c|c|c|c|c|} 
    \hline
    Task & ms ohne ST & mit ST & Delta & Library \\ [0.5ex]
    \hline\hline
    Scripting & 1.111,45 & 1.121,35 & +0,89\% & Redux \\
    \hline
    Painting & 858,05 & 841,05 & -1,98\% & Redux \\
    \hline
    Rendering & 613,25 & 611,05 & -0,36\% & Redux \\
    \hline
    Scripting & 1.680,15 & 1.677,95 & -0,13\% & Pinia \\
    \hline
    Painting & 777,65 & 763,65 & -1,80\% & Pinia \\
    \hline
    Rendering & 651,65 & 642,9 & -1,34\% & Pinia \\
    \hline
    \end{tabular}
  \end{center}
\end{table}

In allen Browsertasks, ausgenommen Scripting bei React, kann ein Rückgang in der Ausführungszeit beobachtet werden. Allerdings ist der Unterschied vernachlässigbar, da dieser sehr gering ist. Auch der Anstieg im Scripting bei React ist mit 0,89\% ebenfalls vernachlässigbar. Siehe (Tab. \ref{tab:performanceAnalysisSTvsNoST})

Die Gerätspezifikation und Versionen der verwendeten Technologien sind aus der Tabelle \ref{tab:specsAndVersions} im Anhang zu entnehmen.

\section{Qualitative Aspekte}

\subsection{Developer Experience}
Die \acrshort{ac:dx} wird durch die zusätzliche Aufgabe der Definition einer Transition Map beeinflusst. Sie führt zu mehr Code und somit zu zusätzlicher Aufwand.

\subsection{Fehleranfälligkeit}
Vorausgesetzt die Transition Map bildet die zulässigen Übergänge vollständig und korrekt ab, kann sich die Applikation nicht in einem unzulässigen Zustand befinden. Obwohl sich hiermit die Fehlerstelle verlegt, ist diese Zentral und nicht an vielen Orten verteilt. Falls sich die Applikation in einem falschen Zustand befindet, ist die Transition Map an einem zentralen Ort zu überprüfen statt die ausgelösten Actions an vielen Orten.

Darüber hinaus bildet die Transition Map die Abläufe in der Applikation ab und kann für eine bessere Nachvollziehbarkeit sorgen. Außerdem ist die Transition Map ein POJO und kann somit ohne weiteres Mocking getestet werden.

\subsection{Lesbarkeit}
Die Lesbarkeit des gesamten Applikationscodes bleibt unverändert, ausgenommen ist die hinzukommende Transition Map. Die Lesbarkeit der Transition Map wird hauptsächlich durch die enthaltene Identitätsfunktionen beeinflusst. Für diese wird der Einsatz von Pure Functions mit geringen Abzweigungen und Funktionsaufrufen empfohlen. Wenn dies eingehalten wird, ist die zyklomatische Komplexität gering, was in der Regel eine bessere Lesbarkeit impliziert. Jedoch hängt die Lesbarkeit stark von den Konventionen und dem Code Style des Authors ab.

\subsection{Wartbarkeit}
Jede hinzukommende Action oder State muss in der Transition Map berücksichtigt werden. Daher steigt die Wartbarkeit.

% TODO Alle Vorteile der einzelnen Libraries bleiben erhalten.