\chapter{Vergleich der Ansätze} \label{vergleich}

% TODO Benchmarking mit Deno 2

Der in vorangegangen Kapiteln vorgestellte Strict Transitions Ansatz erweitert die interne Funktionsweise einer State-Management Lösung. Er erfordert die zusätzliche Definition einer TranisitionMap und es werden leicht geänderte APIs dem Benutzer zur Verfügung gestellt. In diesem Kapitel werden die Standard Stores (unveränderte) zu den mit Strict Transitions verglichen. Es werden die quantifizierbare Kennzahlen Lines of Code und Performance untersucht. Außerdem werden die Aspekte Developer Experience, Fehleranfälligkeit, Wartbarkeit und Lesbarkeit analysiert. Als Basis für den Vergleich dienen die unveränderten Redux und Pinia Stores.

\section{Szenarien}

% Vite Projekte
% React und Vue mit TypeScript
% readme und .gitignore aus einzelnen Projekten gelöscht
% Default Boilerplate, css, demo Counter gelöscht
% prettier installiert, gleiche prettierrc für beide Projekte
% add prettierignore to ignore yaml files, run prettier
% 

% Filterung und Suche

% Data fetching

\section{Lines of Code}

