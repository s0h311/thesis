\chapter{Vergleich der Ansätze} \label{vergleich}

% TODO Benchmarking mit Deno 2

Der in vorangegangen Kapiteln vorgestellte Strict Transitions Ansatz erweitert die interne Funktionsweise einer State-Management Lösung. Er erfordert die zusätzliche Definition einer TranisitionMap und es werden leicht geänderte APIs dem Benutzer zur Verfügung gestellt. In diesem Kapitel werden die Standard Stores (unveränderte) zu den mit Strict Transitions verglichen. Es werden die quantifizierbare Kennzahlen Lines of Code, Bundle Size und Performance untersucht. Außerdem werden die Aspekte Developer Experience, Fehleranfälligkeit, Wartbarkeit und Lesbarkeit analysiert. Als Basis für den Vergleich dienen die unveränderten Redux und Pinia Stores.

\section{Aufbau}

\subsection{Szenario}

Als Grundlage für den Vergleich dient eine Applikation mit einer einzigen Seite, die eine Liste von 194 Produkten beinhaltet. Jedes Produkt besitzt die folgende Struktur:

\begin{lstlisting}
type Product = {
  id: number
  title: string
  description: string
  price: number
  rating: number
  stock: number
  tags: string[]
}  
\end{lstlisting}

Die Applikation fetcht und speichert die Produktdaten im Store. Außerdem können Produkte gefiltert werden. Die Filterkategorien sind: Titel, Preisobergrenze, Mindestbewertung, Verfügbarkeit (stock > 0) und Tags.

\subsection{Eigenschaften der Umgebung und der Applikation}

Die Applikation wurde in React mit Redux und in Vue mit Pinia gebaut. Als Build Tool wurde Vite verwendet. Damit die Ergenisse realitätsnah sind, wurden Production Builds statt Development Builds benutzt und der Applikationbundle über einen Nginx Webserver in einem Docker Container zur Verfügung gestellt. Es wurde ausschließlich Client Side Rendering benutzt.

\begin{center}
  \begin{tabular}{ | m{5cm}| m{5cm} | } 
    \hline
    Gerät & Apple Macbook Pro 2023 mit M3 CPU und 24GB Speicher \\ 
    \hline
    Betriebssystem & MacOS 15.3 \\ 
    \hline
    Docker & 27.4.1 \\ 
    \hline
    OrbStack & 1.9.5 \\ 
    \hline
    Nginx & 1.27.3 \\ 
    \hline
    Chrome Canary & 134.0.6994.0 \\ 
    \hline
    Playwright & 1.50.0 \\ 
    \hline
    Node & 20.17.0 \\ 
    \hline
    Vite & 6.0.5 \\ 
    \hline
    React & 18.3.1 \\ 
    \hline
    Redux & 5.0.1 \\ 
    \hline
    React Redux & 9.2.0 \\ 
    \hline
    Vue & 3.5.13 \\ 
    \hline
    Pinia & 2.3.1 \\ 
    \hline
  \end{tabular}
\end{center}

\subsection{Aufbau des Benchmarks}




% Vite Projekte
% React und Vue mit TypeScript
% readme und .gitignore aus einzelnen Projekten gelöscht
% Default Boilerplate, css, demo Counter gelöscht
% prettier installiert, gleiche prettierrc für beide Projekte
% add prettierignore to ignore yaml files, run prettier
% add analyze script for LOC and file count
% install redux, react-redux and reselect for react
% implement store and components
% install pinia for vue
% implement store and components, same as in react
% provide a mock api data json over nginx and use it in the applications
% foreach project trigger production builds and serve the static assets over nginx (in a  docker container)
% for each application write identical and comprehensive playwright tests 
% Use only Chrome Canary as testing environment
% enable tracing for each test and repeat each 20 times 
% feed each test report to Chrome Canary and note the summary
% copy the source code of Strict Transitions to repository 
% duplicate the react and vue project
% use ST in each of them
% run analyze for each project
% repeat the benchmark with playwright again
% note the summary
% 

\section{Lines of Code}

\section{Bundle Size}

