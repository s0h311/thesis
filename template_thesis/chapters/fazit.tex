\chapter{Fazit} \label{ch:fazit}

\section{Beantwortung der Forschungsfragen}
Die drei zentralen Fragen, mit denen sich die Arbeit beschäftigt sind:

\begin{enumerate}
  \item Können Bugs, die Aufgrund eines falschen Zustandes entstehen, mit Hilfe von Strict Transitions reduziert werden?
  \item Steigt oder sinkt die \acrshort{ac:dx} durch die Einführung von Strict Transitions?
  \item Steigt oder sinkt die Lesbarkeit und Wartbarkeit des Codes durch die Einführung von Strict Transitions?
\end{enumerate}

Der Hauptgegenstände des \acrlong{ac:st} Ansatzes sind die Erhöhung der Produktivität des Entwicklers oder der Entwicklerin und das Erkennen von Bugs in frühen Phasen der Entwicklung im Umgang mit dem Applikationszustand. Um diese Ziele zu erreichen, werden die Zuständsübergänge übersichtlicher an einem zentralen Ort definiert und Laufzeitfehler bei Verstößen geworfen. Die Definition der Zuständsübergänge ist inspiriert von der Übergangsfunktion eines \acrshort{ac:dfa}s.

\section{Ausblick}