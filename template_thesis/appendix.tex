% !TEX root = ./thesis.tex
% appendix example chapter
% @author Thomas Lehmann
%

\chapter{Anhang}

\section{Gerätspezifikation und Versionen der verwendeten Technologien}

\begin{table}[h!]
  \caption{Gerätspezifikation und Versionen der verwendeten Technologien}
  \label{tab:specsAndVersions}
  
  \begin{center}
    \begin{tabular}{ | m{5cm}| m{5cm} | } 
      \hline
      Gerät & Apple Macbook Pro 2023 mit M3 CPU und 24GB Speicher \\ 
      \hline
      Betriebssystem & MacOS 15.3 \\ 
      \hline
      Docker & 27.4.1 \\ 
      \hline
      OrbStack & 1.9.5 \\ 
      \hline
      Nginx & 1.27.3 \\ 
      \hline
      Chrome Canary & 134.0.6994.0 \\ 
      \hline
      Playwright & 1.50.0 \\ 
      \hline
      Node & 20.17.0 \\ 
      \hline
      Vite & 6.0.5 \\ 
      \hline
      React & 18.3.1 \\ 
      \hline
      Redux & 5.0.1 \\ 
      \hline
      React Redux & 9.2.0 \\ 
      \hline
      Vue & 3.5.13 \\ 
      \hline
      Pinia & 2.3.1 \\ 
      \hline
    \end{tabular}
  \end{center}
\end{table}

\section{Beispiele für Implementierung des ST-Ansatzes}

Im Rahmen dieser Arbeit resultierten zwei Implementierung des ST-Ansatzes. Eine für Redux und andere für Pinia. Diese sind im öffentlichen Repository auf Github unter \url{https://github.com/s0h311/strict-transitions} zu finden.

\section{Im Vergleich genutzte Projekte} \label{ap:githubRepository}

Die, im Vergleich genutzten React und Vue Projekte sind im öffentlichen Repository auf Github unter \url{https://github.com/s0h311/strict-transitions-benchmark} zu finden.

\section{Verwendete Hilfsmittel}
In der Tabelle \ref{tab:tooling} sind die im Rahmen der Bearbeitung des Themas der \IthesisKindDE~verwendeten Werkzeuge und Hilfsmittel aufgelistet.

\begin{table}[h!]
\caption{Verwendete Hilfsmittel und Werkzeuge}
\begin{tabular}{|l|l|}
\hline 
\rowcolor{lightgray} Tool & Verwendung \\
\hline
\LaTeX & Textsatz- und Layout-Werkzeug verwendet zur Erstellung dieses Dokuments \\
\hline
OpenAI ChatGPT 4 & Rechtschreib- und Grammatikprüfung \\
\hline
\end{tabular}
\label{tab:tooling}
\end{table}

